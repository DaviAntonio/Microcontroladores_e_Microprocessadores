
%% bare_conf.tex
%% V1.3
%% 2007/01/11
%% by Michael Shell
%% See:
%% http://www.michaelshell.org/
%% for current contact information.
%%
%% This is a skeleton file demonstrating the use of IEEEtran.cls
%% (requires IEEEtran.cls version 1.7 or later) with an IEEE conference paper.
%%
%% Support sites:
%% http://www.michaelshell.org/tex/ieeetran/
%% http://www.ctan.org/tex-archive/macros/latex/contrib/IEEEtran/
%% and
%% http://www.ieee.org/

%%*************************************************************************
%% Legal Notice:
%% This code is offered as-is without any warranty either expressed or
%% implied; without even the implied warranty of MERCHANTABILITY or
%% FITNESS FOR A PARTICULAR PURPOSE! 
%% User assumes all risk.
%% In no event shall IEEE or any contributor to this code be liable for
%% any damages or losses, including, but not limited to, incidental,
%% consequential, or any other damages, resulting from the use or misuse
%% of any information contained here.
%%
%% All comments are the opinions of their respective authors and are not
%% necessarily endorsed by the IEEE.
%%
%% This work is distributed under the LaTeX Project Public License (LPPL)
%% ( http://www.latex-project.org/ ) version 1.3, and may be freely used,
%% distributed and modified. A copy of the LPPL, version 1.3, is included
%% in the base LaTeX documentation of all distributions of LaTeX released
%% 2003/12/01 or later.
%% Retain all contribution notices and credits.
%% ** Modified files should be clearly indicated as such, including  **
%% ** renaming them and changing author support contact information. **
%%
%% File list of work: IEEEtran.cls, IEEEtran_HOWTO.pdf, bare_adv.tex,
%%                    bare_conf.tex, bare_jrnl.tex, bare_jrnl_compsoc.tex
%%*************************************************************************

% *** Authors should verify (and, if needed, correct) their LaTeX system  ***
% *** with the testflow diagnostic prior to trusting their LaTeX platform ***
% *** with production work. IEEE's font choices can trigger bugs that do  ***
% *** not appear when using other class files.                            ***
% The testflow support page is at:
% http://www.michaelshell.org/tex/testflow/



% Note that the a4paper option is mainly intended so that authors in
% countries using A4 can easily print to A4 and see how their papers will
% look in print - the typesetting of the document will not typically be
% affected with changes in paper size (but the bottom and side margins will).
% Use the testflow package mentioned above to verify correct handling of
% both paper sizes by the user's LaTeX system.
%
% Also note that the "draftcls" or "draftclsnofoot", not "draft", option
% should be used if it is desired that the figures are to be displayed in
% draft mode.
%
\documentclass[conference]{IEEEtran}
\usepackage[utf8]{inputenc}
% Escrever em Português do Brasil
\usepackage[brazil]{babel}
% Add the compsoc option for Computer Society conferences.
%
% If IEEEtran.cls has not been installed into the LaTeX system files,
% manually specify the path to it like:
% \documentclass[conference]{../sty/IEEEtran}

\usepackage{graphicx}



% Some very useful LaTeX packages include:
% (uncomment the ones you want to load)


% *** MISC UTILITY PACKAGES ***
%
%\usepackage{ifpdf}
% Heiko Oberdiek's ifpdf.sty is very useful if you need conditional
% compilation based on whether the output is pdf or dvi.
% usage:
% \ifpdf
%   % pdf code
% \else
%   % dvi code
% \fi
% The latest version of ifpdf.sty can be obtained from:
% http://www.ctan.org/tex-archive/macros/latex/contrib/oberdiek/
% Also, note that IEEEtran.cls V1.7 and later provides a builtin
% \ifCLASSINFOpdf conditional that works the same way.
% When switching from latex to pdflatex and vice-versa, the compiler may
% have to be run twice to clear warning/error messages.






% *** CITATION PACKAGES ***
%
%\usepackage{cite}
% cite.sty was written by Donald Arseneau
% V1.6 and later of IEEEtran pre-defines the format of the cite.sty package
% \cite{} output to follow that of IEEE. Loading the cite package will
% result in citation numbers being automatically sorted and properly
% "compressed/ranged". e.g., [1], [9], [2], [7], [5], [6] without using
% cite.sty will become [1], [2], [5]--[7], [9] using cite.sty. cite.sty's
% \cite will automatically add leading space, if needed. Use cite.sty's
% noadjust option (cite.sty V3.8 and later) if you want to turn this off.
% cite.sty is already installed on most LaTeX systems. Be sure and use
% version 4.0 (2003-05-27) and later if using hyperref.sty. cite.sty does
% not currently provide for hyperlinked citations.
% The latest version can be obtained at:
% http://www.ctan.org/tex-archive/macros/latex/contrib/cite/
% The documentation is contained in the cite.sty file itself.






% *** GRAPHICS RELATED PACKAGES ***
%
\ifCLASSINFOpdf
  % \usepackage[pdftex]{graphicx}
  % declare the path(s) where your graphic files are
  % \graphicspath{{../pdf/}{../jpeg/}}
  % and their extensions so you won't have to specify these with
  % every instance of \includegraphics
  % \DeclareGraphicsExtensions{.pdf,.jpeg,.png}
\else
  % or other class option (dvipsone, dvipdf, if not using dvips). graphicx
  % will default to the driver specified in the system graphics.cfg if no
  % driver is specified.
  % \usepackage[dvips]{graphicx}
  % declare the path(s) where your graphic files are
  % \graphicspath{{../eps/}}
  % and their extensions so you won't have to specify these with
  % every instance of \includegraphics
  % \DeclareGraphicsExtensions{.eps}
\fi
% graphicx was written by David Carlisle and Sebastian Rahtz. It is
% required if you want graphics, photos, etc. graphicx.sty is already
% installed on most LaTeX systems. The latest version and documentation can
% be obtained at: 
% http://www.ctan.org/tex-archive/macros/latex/required/graphics/
% Another good source of documentation is "Using Imported Graphics in
% LaTeX2e" by Keith Reckdahl which can be found as epslatex.ps or
% epslatex.pdf at: http://www.ctan.org/tex-archive/info/
%
% latex, and pdflatex in dvi mode, support graphics in encapsulated
% postscript (.eps) format. pdflatex in pdf mode supports graphics
% in .pdf, .jpeg, .png and .mps (metapost) formats. Users should ensure
% that all non-photo figures use a vector format (.eps, .pdf, .mps) and
% not a bitmapped formats (.jpeg, .png). IEEE frowns on bitmapped formats
% which can result in "jaggedy"/blurry rendering of lines and letters as
% well as large increases in file sizes.
%
% You can find documentation about the pdfTeX application at:
% http://www.tug.org/applications/pdftex





% *** MATH PACKAGES ***
%
%\usepackage[cmex10]{amsmath}
% A popular package from the American Mathematical Society that provides
% many useful and powerful commands for dealing with mathematics. If using
% it, be sure to load this package with the cmex10 option to ensure that
% only type 1 fonts will utilized at all point sizes. Without this option,
% it is possible that some math symbols, particularly those within
% footnotes, will be rendered in bitmap form which will result in a
% document that can not be IEEE Xplore compliant!
%
% Also, note that the amsmath package sets \interdisplaylinepenalty to 10000
% thus preventing page breaks from occurring within multiline equations. Use:
%\interdisplaylinepenalty=2500
% after loading amsmath to restore such page breaks as IEEEtran.cls normally
% does. amsmath.sty is already installed on most LaTeX systems. The latest
% version and documentation can be obtained at:
% http://www.ctan.org/tex-archive/macros/latex/required/amslatex/math/





% *** SPECIALIZED LIST PACKAGES ***
%
%\usepackage{algorithmic}
% algorithmic.sty was written by Peter Williams and Rogerio Brito.
% This package provides an algorithmic environment fo describing algorithms.
% You can use the algorithmic environment in-text or within a figure
% environment to provide for a floating algorithm. Do NOT use the algorithm
% floating environment provided by algorithm.sty (by the same authors) or
% algorithm2e.sty (by Christophe Fiorio) as IEEE does not use dedicated
% algorithm float types and packages that provide these will not provide
% correct IEEE style captions. The latest version and documentation of
% algorithmic.sty can be obtained at:
% http://www.ctan.org/tex-archive/macros/latex/contrib/algorithms/
% There is also a support site at:
% http://algorithms.berlios.de/index.html
% Also of interest may be the (relatively newer and more customizable)
% algorithmicx.sty package by Szasz Janos:
% http://www.ctan.org/tex-archive/macros/latex/contrib/algorithmicx/




% *** ALIGNMENT PACKAGES ***
%
%\usepackage{array}
% Frank Mittelbach's and David Carlisle's array.sty patches and improves
% the standard LaTeX2e array and tabular environments to provide better
% appearance and additional user controls. As the default LaTeX2e table
% generation code is lacking to the point of almost being broken with
% respect to the quality of the end results, all users are strongly
% advised to use an enhanced (at the very least that provided by array.sty)
% set of table tools. array.sty is already installed on most systems. The
% latest version and documentation can be obtained at:
% http://www.ctan.org/tex-archive/macros/latex/required/tools/


%\usepackage{mdwmath}
%\usepackage{mdwtab}
% Also highly recommended is Mark Wooding's extremely powerful MDW tools,
% especially mdwmath.sty and mdwtab.sty which are used to format equations
% and tables, respectively. The MDWtools set is already installed on most
% LaTeX systems. The lastest version and documentation is available at:
% http://www.ctan.org/tex-archive/macros/latex/contrib/mdwtools/


% IEEEtran contains the IEEEeqnarray family of commands that can be used to
% generate multiline equations as well as matrices, tables, etc., of high
% quality.


%\usepackage{eqparbox}
% Also of notable interest is Scott Pakin's eqparbox package for creating
% (automatically sized) equal width boxes - aka "natural width parboxes".
% Available at:
% http://www.ctan.org/tex-archive/macros/latex/contrib/eqparbox/





% *** SUBFIGURE PACKAGES ***
%\usepackage[tight,footnotesize]{subfigure}
% subfigure.sty was written by Steven Douglas Cochran. This package makes it
% easy to put subfigures in your figures. e.g., "Figure 1a and 1b". For IEEE
% work, it is a good idea to load it with the tight package option to reduce
% the amount of white space around the subfigures. subfigure.sty is already
% installed on most LaTeX systems. The latest version and documentation can
% be obtained at:
% http://www.ctan.org/tex-archive/obsolete/macros/latex/contrib/subfigure/
% subfigure.sty has been superceeded by subfig.sty.



%\usepackage[caption=false]{caption}
%\usepackage[font=footnotesize]{subfig}
% subfig.sty, also written by Steven Douglas Cochran, is the modern
% replacement for subfigure.sty. However, subfig.sty requires and
% automatically loads Axel Sommerfeldt's caption.sty which will override
% IEEEtran.cls handling of captions and this will result in nonIEEE style
% figure/table captions. To prevent this problem, be sure and preload
% caption.sty with its "caption=false" package option. This is will preserve
% IEEEtran.cls handing of captions. Version 1.3 (2005/06/28) and later 
% (recommended due to many improvements over 1.2) of subfig.sty supports
% the caption=false option directly:
%\usepackage[caption=false,font=footnotesize]{subfig}
%
% The latest version and documentation can be obtained at:
% http://www.ctan.org/tex-archive/macros/latex/contrib/subfig/
% The latest version and documentation of caption.sty can be obtained at:
% http://www.ctan.org/tex-archive/macros/latex/contrib/caption/




% *** FLOAT PACKAGES ***
%
%\usepackage{fixltx2e}
% fixltx2e, the successor to the earlier fix2col.sty, was written by
% Frank Mittelbach and David Carlisle. This package corrects a few problems
% in the LaTeX2e kernel, the most notable of which is that in current
% LaTeX2e releases, the ordering of single and double column floats is not
% guaranteed to be preserved. Thus, an unpatched LaTeX2e can allow a
% single column figure to be placed prior to an earlier double column
% figure. The latest version and documentation can be found at:
% http://www.ctan.org/tex-archive/macros/latex/base/



%\usepackage{stfloats}
% stfloats.sty was written by Sigitas Tolusis. This package gives LaTeX2e
% the ability to do double column floats at the bottom of the page as well
% as the top. (e.g., "\begin{figure*}[!b]" is not normally possible in
% LaTeX2e). It also provides a command:
%\fnbelowfloat
% to enable the placement of footnotes below bottom floats (the standard
% LaTeX2e kernel puts them above bottom floats). This is an invasive package
% which rewrites many portions of the LaTeX2e float routines. It may not work
% with other packages that modify the LaTeX2e float routines. The latest
% version and documentation can be obtained at:
% http://www.ctan.org/tex-archive/macros/latex/contrib/sttools/
% Documentation is contained in the stfloats.sty comments as well as in the
% presfull.pdf file. Do not use the stfloats baselinefloat ability as IEEE
% does not allow \baselineskip to stretch. Authors submitting work to the
% IEEE should note that IEEE rarely uses double column equations and
% that authors should try to avoid such use. Do not be tempted to use the
% cuted.sty or midfloat.sty packages (also by Sigitas Tolusis) as IEEE does
% not format its papers in such ways.





% *** PDF, URL AND HYPERLINK PACKAGES ***
%
%\usepackage{url}
% url.sty was written by Donald Arseneau. It provides better support for
% handling and breaking URLs. url.sty is already installed on most LaTeX
% systems. The latest version can be obtained at:
% http://www.ctan.org/tex-archive/macros/latex/contrib/misc/
% Read the url.sty source comments for usage information. Basically,
% \url{my_url_here}.





% *** Do not adjust lengths that control margins, column widths, etc. ***
% *** Do not use packages that alter fonts (such as pslatex).         ***
% There should be no need to do such things with IEEEtran.cls V1.6 and later.
% (Unless specifically asked to do so by the journal or conference you plan
% to submit to, of course. )


% correct bad hyphenation here
\hyphenation{op-tical net-works semi-conduc-tor}


\begin{document}
%
% paper title
% can use linebreaks \\ within to get better formatting as desired
\title{Redu\c{c}\~ao da Bradicinesia em Pacientes com Mal de Parkinson em Estado Inicial}




% author names and affiliations
% use a multiple column layout for up to three different
% affiliations
\author{\IEEEauthorblockN{Davi Ant\^onio da Silva Santos}
\IEEEauthorblockA{Graduando em Engenharia Eletr\^onica\\
Universidade de Bras\'ilia\\
Gama, Brasil\\
Email: wokep.ma.wavid@gmail.com}
\and
\IEEEauthorblockN{Victor Aguiar Coutinho}
\IEEEauthorblockA{Graduando em Engenharia Eletr\^onica\\
	Universidade de Bras\'ilia\\
	Gama, Brasil\\
	Email: victor.a.coutinho@gmail.com}}

% conference papers do not typically use \thanks and this command
% is locked out in conference mode. If really needed, such as for
% the acknowledgment of grants, issue a \IEEEoverridecommandlockouts
% after \documentclass

% for over three affiliations, or if they all won't fit within the width
% of the page, use this alternative format:
% 
%\author{\IEEEauthorblockN{Michael Shell\IEEEauthorrefmark{1},
%Homer Simpson\IEEEauthorrefmark{2},
%James Kirk\IEEEauthorrefmark{3}, 
%Montgomery Scott\IEEEauthorrefmark{3} and
%Eldon Tyrell\IEEEauthorrefmark{4}}
%\IEEEauthorblockA{\IEEEauthorrefmark{1}School of Electrical and Computer Engineering\\
%Georgia Institute of Technology,
%Atlanta, Georgia 30332--0250\\ Email: see http://www.michaelshell.org/contact.html}
%\IEEEauthorblockA{\IEEEauthorrefmark{2}Twentieth Century Fox, Springfield, USA\\
%Email: homer@thesimpsons.com}
%\IEEEauthorblockA{\IEEEauthorrefmark{3}Starfleet Academy, San Francisco, California 96678-2391\\
%Telephone: (800) 555--1212, Fax: (888) 555--1212}
%\IEEEauthorblockA{\IEEEauthorrefmark{4}Tyrell Inc., 123 Replicant Street, Los Angeles, California 90210--4321}}




% use for special paper notices
%\IEEEspecialpapernotice{(Invited Paper)}




% make the title area
\maketitle


\begin{abstract}
%\boldmath
O uso repetitivo das mãos com auxílio audiovisual tem resultados positivos em pessoas diagnosticadas com Mal de Parkinson em estado inicial. Tem-se como objetivo apresentar um sistema a base de MSP430 fácil de utilizar que execute jogos interativos. Quem utilizá-lo terá redução da bradicinesia, principal sintoma da doença. 
\end{abstract}
\begin{IEEEkeywords}
	Mal de Parkinson; tratamento; exercício das mãos;  MSP430.
\end{IEEEkeywords}

% IEEEtran.cls defaults to using nonbold math in the Abstract.
% This preserves the distinction between vectors and scalars. However,
% if the conference you are submitting to favors bold math in the abstract,
% then you can use LaTeX's standard command \boldmath at the very start
% of the abstract to achieve this. Many IEEE journals/conferences frown on
% math in the abstract anyway.

% no keywords




% For peer review papers, you can put extra information on the cover
% page as needed:
% \ifCLASSOPTIONpeerreview
% \begin{center} \bfseries EDICS Category: 3-BBND \end{center}
% \fi
%
% For peerreview papers, this IEEEtran command inserts a page break and
% creates the second title. It will be ignored for other modes.
\IEEEpeerreviewmaketitle



\section{Introdução}
% no \IEEEPARstart

% You must have at least 2 lines in the paragraph with the drop letter
% (should never be an issue)


\subsection{Justificativa}
O uso de exercícios simples com as mãos, repetição de sequências acompanhados de estímulos audiovisuais pode ser usado em conjunto com os tratamentos tradicionais para auxiliar na reabilitação dos pacientes acometidos do mal de Parkinson em estado inicial, atenuando a bradicinesia, um dos principais sintomas da doença \cite{medtronic}, caracterizado pela extrema lentidão dos movimentos \cite{pelosin}.

Assim, baseado no equipamento utilizado no artigo de Elisa Pelosin, propõe-se uma solução embarcada de baixo custo baseada no microcontrolador MSP430, utilizando-se a respectiva placa de desenvolvimento, a LaunchPad.




\subsection{Objetivos}
Exercício tem um forte impacto na recuperação de pacientes diagnosticados com Mal de Parkinson, porém um dos maiores problemas é a falta de interesses dos pacientes em exercícios físicos \cite{playford}. Como Diane Playford explicita em seu artigo Exercise and Parkinson’s disease, o desafio não é definir programas de atividades saudáveis, mas incentivar pessoas a encontrarem atividades em que elas achem proveitosas.

Tem-se como objetivo apresentar um projeto de dispositivo que incentive aos pacientes diagnosticados com o Mal de Parkinson em estado inicial a exercitar as mãos como parte de tratamento para não perder o controle dos movimentos.

Com isso, almeja-se colocar jogos interativos de forma a interessar os pacientes a manterem o tratamento. Jogos que serão feitos a base de pesquisa em tratamentos de pacientes de Mal de Parkinson.


\subsection{Requisitos}
Com base no público alvo, pacientes diagnosticados com Parkinson em estado inicial, e o objetivo do sistema, melhorar a coordenação motora ao atenuar a bradicinesia, foram elencados os seguintes requisitos:
\begin{itemize}[\IEEEsetlabelwidth{5}]
	\item O sistema proposto deve ser de fácil construção e baixo custo;
	\item O sistema deve ser fácil de operar;
	\item O sistema deve fornecer estímulos audiovisuais;
	\item A interface deve ser de fácil compreensão;
	\item Sequências geradas devem ser aleatórias, evitando que o paciente decore-as.
\end{itemize}



\subsection{Benefícios}
O sistema embarcado proposto, conforme exposto anteriormente, reduz a bradicinesia através da repetição de sequências reforçadas por estímulos audiovisuais, em conjunto com os tratamentos tradicionais, em pacientes diagnosticados com Parkinson em estado inicial.

\section{Descrição do sistema}

A proposta é apresentar um dispositivo que auxilia na redução da bradicinesia e que, de alguma forma, incentive quem a possui a exercitar as mãos. Para isso, o dispositivo possui um jogo que soa sinais com frequências diferentes e o usuário deva apertar os botões correspondentes. Importante ressaltar que o usuário deve seguir o protocolo de uso para melhores resultados.

Na inicialização do sistema, são emitidos quatro estímulos sonoros, correspondentes às frequências que serão utilizadas durante o jogo. Em seguida, é tocada sequência, a qual inicia em um elemento, e acende-se o LED verde enquanto o número de elementos da sequência não for digitado nos botões, indicando que o sistema está esperando uma entrada do usuário. Ao terminar a sequência, o LED verde desliga.

Caso o usuário acerte a sequência, o LED verde pisca, indicando o acerto, e o nível é incrementado, isto é, a próxima sequência aumentará em um elemento, colocado ao fim da sequência anterior. Também, em caso de acerto da sequência, o sistema esperará uma entrada do usuário indicando se ele quer continuar o jogo, função atribuída ao botão \emph{S1}, ou se ele quer desistir do jogo, função do botão \emph{S2}. Esse momento é expresso ao usuário com ambos os LEDs ligados. Caso o usuário queira continuar, o sistema mostrará a próxima sequência, e caso ele queira desistir, o jogo desligará e exigirá um \emph{reset} para reiniciar.

Caso o usuário erre, o LED vermelho piscará indicando um erro na digitação e, em seguida, os dois LEDs acenderão, esperando que o usuário informe que quer continuar jogando, intenção atribuída ao botão \emph{S1}, ou que quer desistir do jogo, atribuída a \emph{S2}. Caso o usuário deseje continuar, o jogo reiniciará, mas caso desista, o jogo desligará e será necessário um \emph{reset} para reiniciar.

\section{Descrição do Hardware}

\begin{figure}[!t]
	\centering 
	\includegraphics[width=2.5in]{Esquematico.png} 
	\caption{Esquemático do dispositivo proposto}
	\label{esq} 
\end{figure}

O projeto é composto pela placa de desenvolvimento Launchpad, a qual contém um microcontrolador MSP430G2553; quatro \emph{pushbuttons} para as entradas do usuário, \emph{S1}, \emph{S2}, \emph{S3}, \emph{S4}, conectados às entradas \emph{P1.3}, \emph{P1.4}, \emph{P1.5} e \emph{P1.7}, respectivamente; dois LEDs, sendo um verde, conectado à \emph{P1.6} e outro vermelho, ligado à \emph{P2.0}; um \emph{buzzer} eletromagnético; um transistor NPN BC337-25, conectado à \emph{P2.4}; e resistores de 10 k$\Omega$, 15 k$\Omega$, 1 k$\Omega$ e 100 $\Omega$ para limitar a corrente fornecida pelas portas do MSP.

Os quatro \emph{pushbuttons}, estão conectados diretamente ao Vcc = 3,5 V . Utilizou-se esta configuração pois no código foram habilitados os resistores internos de \emph{pull-down} no microcontrolador.

Os resistores colocados sobre os LEDs limitam a corrente entregue pelas portas digitais do MSP, que não deve ser maior que 6 mA em módulo, segundo \cite{datamsp}. Já os resistores colocados antes do transistor limitam a corrente entregue pela porta digital do MSP e da fonte de 3,5 V ao \emph{buzzer}.

Os resistores conectados ao transistor foram dimensionados segundo as especificações do MSP, do \emph{buzzer} e do próprio transistor. Usando um multímetro digital, determinou-se que o \emph{hfe} do transistor era de 308, o que estava dentro dos limites estabelecidos no \emph{datasheet} do mesmo, entre 160 e 400 \cite{databc337}.

O \emph{datasheet} do \emph{buzzer} indicava uma corrente máxima de 40 mA \cite{databuzzer}, que no transistor corresponderia à corrente no coletor, e este dado foi usado para se calcular a corrente na base do transistor, 129,870 $\mu$A. A resistência na base foi dimensionada através da equação Rb = (Vcc - Vbe)/Ib. Considerou-se Vbe = 0,7 V e Vcc = 3,5 V, e obteve-se uma resistência de cerca de 21 k$\Omega$, a qual foi substituída por uma de 25 k$\Omega$.


\section{Descrição do Software}

\subsection{Software no Energia}

A plataforma de prototipagem utilizada para codificar, compilar e enviar para o MSP430 foi o Energia 1.6.10E18. No código primeiramente, definiu variáveis para valores constantes e portas. A \emph{TSMMS} é uma constante que representa o tempo que o \emph{buzzer} fica ligado ou desligado pois a taxa entre sons que apresenta melhores resultados é de 3 Hz \cite{pelosin}. A \emph{LEDverde} representa o pino em que o LED vermelho está conectado, no caso o \emph{P1.6} do MSP430. Já a constante \emph{LEDvermelho} representa o pino \emph{P2.0} na qual o LED vermelho está conectado. A \emph{PPWM} é o pino onde o \emph{buzzer} está conectado.

O software foi dividido em funções: \emph{ler\_portas}, \emph{toca\_buzzer}, \emph{pisca2x}, \emph{setup} e \emph{loop}. A primeira função no código é a \emph{ler\_portas}. Essa tem como saída o valor do botão no qual foi apertado, sendo o botão \emph{S1} representando o sinal com mais baixa frequência e o \emph{S4} o sinal com mais alta frequência. A função começa com um \emph{while(1)} pois ele só deve sair do laço quando algum botão for apertado e os demais não. Considerou os botões configurados como \emph{pull down}. Para o usuário saber que que o botão foi apertado o LED verde, como pode ser visto na função principal \emph{loop}, que anteriormente estava ligado desliga. O \emph{delay} de 100 milissegundos para dar tempo do usuário soltar o botão.

A função \emph{toca\_buzzer} tem como entrada \emph{nLED} e dentro da função tem o vetor com frequências já definidas. É tocado o \emph{buzzer} na frequência que está na posição $\emph{nLED -1}$ do vetor \emph{frequencias}. Após tocado por \emph{TSOMMS} ms o buzzer desliga com a função \emph{noTone} por \emph{TSOMMS} ms.

Para indicar que uma pessoa acertou ou errou um LED pisca duas vezes, sendo que se o usuário acertou o LED verde que pisca, caso contrário o LED vermelho que pisca. Para isso, desenvolveu a função \emph{pisca2x} tendo como entrada o LED desejado \emph{LED\_que\_pisca}. Os \emph{delays} de 100 ms foi definido para que o usuário perceba que piscou e não confunda com, no caso do LED verde, o momento para digitar. O \emph{delay} de 1000 ms é para o usuário perceber a parte de piscar indicando acerto ou erro e a parte que liga os LEDs para o usuário decidir se quer continuar ou não, essas duas condições será explicada posteriormente.

É definido as portas de entrada e saídas na função \emph{setup}, padrão no Energia. Definiu os pinos \emph{P1.3}, \emph{P1.4}, \emph{P1.5} e \emph{P1.7} como entrada e \emph{pull down} para leitura dos botões \emph{S1}, \emph{S2}, \emph{S3} e \emph{S4}, respectivamente. Definiu os pinos \emph{P2.4} (representando o buzzer), \emph{LEDvermelho} e emph{LEDverde} como saída. Após essas definições colocou todas as saídas em nível lógico alto e gera um número aleatório a partir da leitura da temperatura do ambiente no \emph{A10}. Por fim habilita a entrada e saída serial da placa.

A função principal é \emph{loop}. Começa zerando todas as variáveis, exceto o \emph{nivel} já que o primeiro nível considerado foi o zero. Ele apresenta todas as frequências no vetor na ordem dos botões que o representam. Dentro do \emph{while(1)} está o jogo em si. A variavel \emph{continuar\_jogo} igual a zero significa que o jogo pode continuar, essa condição é avaliada no final do jogo. Considerou-se o nível máximo de 32, caso o nível seja esse então o jogo finaliza.

A posição do vetor de frequências é gerada por meio da função \emph{random} e colocada na posição \emph{i} da sequência de posições. Logo após é tocado as frequências desde  primeira até a atual gerada com a função \emph{toca\_buzzer}. É feito a leitura dos botões com a função \emph{ler\_portas()} e decide se o usuário acertou a sequência comparando os vetores \emph{seq\_gen} (sequência gerada) e \emph{seq\_read} (sequência dos botões).

Se a sequência digitada não for igual ao apresentada então a variável \emph{fail} fica igual a um. Com o fail igual a um, o LED vermelho pisca duas vezes indicando o erro e o nível retorna para um. Caso contrário, ou seja acertou, o LED verde pisca duas vez para indicar o acerto o nível aumenta e a variável \emph{i} também aumenta para quando retornar o \emph{while} a nova posição do vetor das frequências ser guardada na próxima posição do vetor \emph{seq\_gen}. 

Como a decisão de se o usuário acertou ou não, chega na etapa de decidir se o mesmo deseja continuar a jogar. Para isso, ambos os LEDs ficam ligados e é feito a leitura dos botões, essa leitura é armazenada na variável \emph{continuar\_jogo}. Se apertou \emph{S1} significa que deseja continuar e o \emph{S2} para finalizar o jogo. Se for continuar, os LEDs desligam e o LED verde pisca duas vezes indicando o desejo do usuário. Se não deseja continuar, o LED vermelho pisca duas vezes indicando o desejo de não continuar e o programa entra em um laço infinito (a linha do \emph{while(1);}).


%\subsection{Linguagem C e Assembly}
%asd
%\section{Resultados}
%asd
%\section{Conclusão}
%asdf


% An example of a floating figure using the graphicx package.
% Note that \label must occur AFTER (or within) \caption.
% For figures, \caption should occur after the \includegraphics.
% Note that IEEEtran v1.7 and later has special internal code that
% is designed to preserve the operation of \label within \caption
% even when the captionsoff option is in effect. However, because
% of issues like this, it may be the safest practice to put all your
% \label just after \caption rather than within \caption{}.
%
% Reminder: the "draftcls" or "draftclsnofoot", not "draft", class
% option should be used if it is desired that the figures are to be
% displayed while in draft mode.
%
%\begin{figure}[!t]
%\centering
%\includegraphics[width=2.5in]{myfigure}
% where an .eps filename suffix will be assumed under latex, 
% and a .pdf suffix will be assumed for pdflatex; or what has been declared
% via \DeclareGraphicsExtensions.
%\caption{Simulation Results}
%\label{fig_sim}
%\end{figure}

% Note that IEEE typically puts floats only at the top, even when this
% results in a large percentage of a column being occupied by floats.


% An example of a double column floating figure using two subfigures.
% (The subfig.sty package must be loaded for this to work.)
% The subfigure \label commands are set within each subfloat command, the
% \label for the overall figure must come after \caption.
% \hfil must be used as a separator to get equal spacing.
% The subfigure.sty package works much the same way, except \subfigure is
% used instead of \subfloat.
%
%\begin{figure*}[!t]
%\centerline{\subfloat[Case I]\includegraphics[width=2.5in]{subfigcase1}%
%\label{fig_first_case}}
%\hfil
%\subfloat[Case II]{\includegraphics[width=2.5in]{subfigcase2}%
%\label{fig_second_case}}}
%\caption{Simulation results}
%\label{fig_sim}
%\end{figure*}
%
% Note that often IEEE papers with subfigures do not employ subfigure
% captions (using the optional argument to \subfloat), but instead will
% reference/describe all of them (a), (b), etc., within the main caption.


% An example of a floating table. Note that, for IEEE style tables, the 
% \caption command should come BEFORE the table. Table text will default to
% \footnotesize as IEEE normally uses this smaller font for tables.
% The \label must come after \caption as always.
%
%\begin{table}[!t]
%% increase table row spacing, adjust to taste
%\renewcommand{\arraystretch}{1.3}
% if using array.sty, it might be a good idea to tweak the value of
% \extrarowheight as needed to properly center the text within the cells
%\caption{An Example of a Table}
%\label{table_example}
%\centering
%% Some packages, such as MDW tools, offer better commands for making tables
%% than the plain LaTeX2e tabular which is used here.
%\begin{tabular}{|c||c|}
%\hline
%One & Two\\
%\hline
%Three & Four\\
%\hline
%\end{tabular}
%\end{table}


% Note that IEEE does not put floats in the very first column - or typically
% anywhere on the first page for that matter. Also, in-text middle ("here")
% positioning is not used. Most IEEE journals/conferences use top floats
% exclusively. Note that, LaTeX2e, unlike IEEE journals/conferences, places
% footnotes above bottom floats. This can be corrected via the \fnbelowfloat
% command of the stfloats package.








% conference papers do not normally have an appendix


% use section* for acknowledgement






% trigger a \newpage just before the given reference
% number - used to balance the columns on the last page
% adjust value as needed - may need to be readjusted if
% the document is modified later
%\IEEEtriggeratref{8}
% The "triggered" command can be changed if desired:
%\IEEEtriggercmd{\enlargethispage{-5in}}

% references section

% can use a bibliography generated by BibTeX as a .bbl file
% BibTeX documentation can be easily obtained at:
% http://www.ctan.org/tex-archive/biblio/bibtex/contrib/doc/
% The IEEEtran BibTeX style support page is at:
% http://www.michaelshell.org/tex/ieeetran/bibtex/
%\bibliographystyle{IEEEtran}
% argument is your BibTeX string definitions and bibliography database(s)
%\bibliography{IEEEabrv,../bib/paper}
%
% <OR> manually copy in the resultant .bbl file
% set second argument of \begin to the number of references
% (used to reserve space for the reference number labels box)
\begin{thebibliography}{1}



  
  \bibitem{pelosin}
  PELOSIN, ~E. et al., Reduction of Bradykinesia of Finger Movements by a Single Session of Action Observation in Parkinson Disease.\hskip 1em plus
  0.5em minus 0.4em \emph{Neurorehabilitation and Neural Repair}. \relax p.552-560, 2013.
  
  \bibitem{medtronic}
  Medtronic Brasil. \emph{Sobre a Doença de Parkinson}. Disponível em: \textless http:www.medtronicbrasil.com.br/your-health/parkinsons-disease/index.htm \textgreater. Acesso em 03 de setembro de 2017.
  
  \bibitem{playford}
  PLAYFORD, ~Diane. Exercise and Parkinson’s disease. \emph{Neurol Neurosurg Psychiatry},\hskip 1em plus
  0.5em minus 0.4em\relax 2011;82:1185.
  
  \bibitem{databc337}
  FAIRCHILD SEMICONDUCTOR CORPORATION. \emph{BC337-25 NPN General Purpose Amplifier}. 1997.
  
  \bibitem{databuzzer}
  \emph{KX-1200 Series Magnetic Transducer}
  
  \bibitem{datamsp}
  TEXAS INSTRUMENTS.\emph{MSP430G2x53, MSP430G2x13 Mixed Signal Microcontroller}. p.70, 2011.
  

\end{thebibliography}




% that's all folks
\end{document}


